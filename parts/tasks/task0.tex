\subsection{Задание 1}
\subsubsection{Условие}
Докажите теоретически и проверьте экспериментально неравенство
\[\frac{1}{n}M(A) \leq ||A||_2 \leq M(A) \text{, где } M(A) = n \times \underbrace{\max}_{1 \leq i,j \leq n} |a_{ij}|\]
\subsubsection{Решение}
Докажем неравенство теоретически:

\begin{enumerate}
\item
  Верхняя граница \(\|A\|_2 \leq M(A)\):
Обозначим через \(\lambda\) наибольшее собственное значение матрицы \(A^*A\), где \(A^*\) - это эрмитово сопряжение матрицы \(A\).
Тогда \( \|A\|_2 = \sqrt{\lambda} \). С учетом этого и свойств собственных значений, можно получить,
что \( \|A\|_2^2 =\lambda \leq \max( \text{собственные значения}(A^*A)) = M(A)^2 \). Таким образом, верно, что \( \|A\|_2 \leq M(A) \).
\item
Нижняя граница \(\frac{1}{n}M(A) \leq \|A\|_2\):
Обозначим через \(\lambda_{\text{min}}\) - наименьшее собственное значение матрицы \(A^*A\).
Тогда \( \frac{1}{n}M(A)^2 \leq \frac{1}{n} \max(\text{собственные}\)
\(\text{значения}(A^*A))=\frac{1}{n} \max(\lambda) \leq \lambda_{\text{max}} \leq \|\|A\|_2 \|\|^2_2 = \|A\|^2_2 \).
Следовательно \(\frac{1}{n}M(A) \leq \|A\|_2\).
\end{enumerate}

Теперь проверим это неравенство с помощью кода на Python, используя библиотеку "numPy": 

\begin{figure}[H]
  \lstinputlisting[
      style=rust
  ]{src/tasks/task_0.py}
  \caption*{Код}
\end{figure}

Этот код создает случайную матрицу A, вычисляет спектральную норму и значение M(A), затем проверят неравенство на основе этих значений.
