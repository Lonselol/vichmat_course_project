\section{Введение}
Объектом исследования являются численные методы решения задач 
линейной алгебры, а также программное обеспечение, реализующее эти 
методы.

Цель работы – ознакомиться с численными методами решения систем 
линейных алгебраических уравнений, нахождения обратных матриц, 
решения проблемы собственных значений, решить предложенные типовые 
задачи, сформулировать выводы по полученным решениям, отметить 
достоинства и недостатки методов, сравнить удобство использования и 
эффективность работы каждой использованной программы, приобрести 
практические навыки и компетенции, а также опыт самостоятельной 
профессиональной деятельности, а именно:

\begin{enumerate}
  \item создать алгоритм решения поставленной задачи и реализовать его, 
  протестировать программы;
  \item освоить теорию вычислительного эксперимента; современных 
  компьютерных технологий; 
  \item приобрести навыки представления итогов проделанной работы в виде 
  отчета, оформленного в соответствии с имеющимися требованиями, с 
  привлечением современных средств редактирования и печати.
\end{enumerate}

  Работа над курсовым проектом предполагает выполнение следующих 
задач:

\begin {enumerate}
  \item дальнейшее углубление теоретических знаний обучающихся и их 
  систематизацию;
  \item получение и развитие прикладных умений и практических навыков по 
  направлению подготовки;
  \item овладение методикой решения конкретных задач;
  \item развитие навыков самостоятельной работы;
  \item развитие навыков обработки полученных результатов, анализа и 
  осмысления их с учетом имеющихся литературных данных;
  \item приобретение навыков оформления описаний программного продукта;
  \item повышение общей и профессиональной эрудиции.
\end {enumerate}
