\subsection{Анализ методов}
\subsubsection{Тестирование}
Для тестирования была заранее подготовлена 1 матрица с низким числом обусловленности и сгенерировано несколько дополнительных.
В начале программа считает решение встроенной функцией библиотеки "numPy", а после ищет его всеми реализованными методами. Были получены следующие результаты:
\lstinputlisting[
  style=txt
]{tests.txt}
\subsubsection{Погрешность}
Погрешность вычислений будем считать по следующей формуле:
\[
  \frac{||X_{\text{точное}}-X_{\text{полученное}}||}{||X_{\text{точное}}||}
\]
Рассмотренные методы имеют сложность \(O(N)\). Это линейная сложность, но, несмотря на это, если мы имеем большое число обусловленности (больше 5), то появляется серьёзная погрешность.