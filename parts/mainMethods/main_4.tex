\externaldocument{main_0}
\subsection{Метод немонотонной прогонки}
Алгоритм состоит из двух шагов: прямой и обратных ход:
\subsubsection{Прямой ход}
На этом этапе система \ref{eq:1_1} приводится к двухдиагональному виду и вычисляются прогоночные коэффициенты.

\begin{enumerate}
  \item Пусть \( \bar{\alpha_1}=B_1, \bar{\beta_1}=C_1, \bar{\gamma}=F_1\) (первое уравнение системы \ref{eq:1_1} имеет двухчленный вид)
  \item Тогда первое уравнение примет вид:
        \[\alpha_1 U_1 + \beta_1 U_2 = \gamma_1, \text{где} \alpha_1=\frac{\bar{\alpha_1}}{\lambda_1}, 
        \beta_1=\frac{\bar{\beta_1}}{\lambda_1}, \gamma_1=\frac{\bar{\gamma_1}}{\lambda_1}, 
        \lambda_1=\text{max(\( |\bar{\alpha_1}| |\bar{\beta_1}| |\bar{\gamma_1}| \))}\]
        Т.к. коэффициенты \(\alpha_1, \beta_1, \gamma_1\) определены с точностью до произвольного множителя, то мы делим их на \(\lambda_1\), чтобы по модулю они не были больше единицы.
  \item Пусть \((k-1)\)-е уравнение имеет двухчленный вид \label{eq:5_1} \[\alpha_{k-1}U_{k-1}+\beta_{k-1}U_k=\gamma_{k-1}\]
  \item Уберём из \(k\)-го уравнения системы \ref{eq:1_1} \(U_{k-1}\):
        \[A_k \left( -\frac{\beta_{k-1}}{\alpha_{k-1}}U_k + \frac{\gamma_{k-1}}{\alpha_{k-1}} \right) + B_k U_k + C_k U_{k+1} = F_k\]
  \item Отсюда получаем: \[\left(B_k-A_k \frac{\beta_{k-1}}{\alpha_{k-1}}\right) U_k+C_k U_{k+1}=F_k-A_k \frac{\gamma_{k-1}}{\alpha_{k-1}}\]
  \item Домножаем на \(\alpha_{k-1}\) и получаем \[\left(B_k \alpha_{k-1}-A_k \beta_{k-1} \right) U_k+\alpha_{k-1} C_k U_{k+1}=F_k \alpha_{k-1}-A_k \gamma_{k-1}\]
  \item Пусть: \label{eq:5_2}
        \[\bar{\alpha_k} = \alpha_{k-1}B_k - \beta_{k-1}A_k\]
        \[\bar{\beta_k} = \alpha_{k-1}C_k\]
        \[\bar{\gamma_k} = \alpha_{k-1}F_k - \gamma_{k-1}A_k\]
  \item После нормировки получим соотношение: \[\alpha_k U_k + \beta_k U_{k+1} = \gamma_k\] \label{eq:5_3}
  \item Где \label{eq:5_4}\[
        \alpha_1=\frac{\bar{\alpha_1}}{\lambda_1}, 
        \beta_1=\frac{\bar{\beta_1}}{\lambda_1},
        \gamma_1=\frac{\bar{\gamma_1}}{\lambda_1}, 
        \lambda_1=\text{max(\( |\bar{\alpha_1}| |\bar{\beta_1}| |\bar{\gamma_1}| \))}
        \]
\end{enumerate}
Строим прогоночные коэффициенты для \(k=2,3,...,N-1\) по формулам из П.\ref{eq:5_2} и П.\ref{eq:5_4}

\subsubsection{Обратный ход}
\(N\)-е уравнение системы \ref{eq:1_1} не участвовало на этапе прямого хода.
Пусть \(k=N-1\). Того оно вместе с соотношением из П.\ref{eq:5_2} образует систему:
\[
  \begin{cases}
    A_N U_{N-1} + B_N U_N = F_N,
    \\
    \alpha_{N-1} U_{N-1} + \beta_{N-1} U_N = \gamma_{N-1}
  \end{cases}  
\]
Из этой системы найдём значения \(U_N\) и \(U_{N-1}\):
\[
  U_N = \frac{A_N \gamma_{N-1} - \alpha_{N-1} F_N}{A_N\beta_{N-1}-\alpha_{N-1}B_N}
\]
\[
  U_{N-1} = \frac{F_N\beta_{N-1}-\gamma_{N-1}B_N}{A_N \beta_{N-1} - \alpha_{N-1}B_N}
\]
Рассматривая матрицу системы из П.\ref{eq:5_3}, можно выделить 3 блока:
\begin{enumerate}
  \item Строки, образующие блоки, для которых \(|\bar{\alpha_i}| \geq |\bar{\beta_i}|\)
  \item Строки, образующие блоки, для которых \(0<|\alpha_i|<|\beta_i|\). 
  \item Строки, в которых \(\alpha_i=0\)
\end{enumerate}

Для блоков первого типа вычисляем значения искомого решения по формулам из П.\ref{eq:5_3}.
В блоках третьего типа с помощью соответствующих уравнений системы \ref{eq:1_1}.
Если \(a_k=0, k=i\), то \(U_i=\frac{F_{i+1}-B_{i+1}U_{i+1}-C_{i+1}U_{i+2}}{A_{i+1}}\)
