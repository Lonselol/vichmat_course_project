\externaldocument{main_0}
\subsection{Метод монотонной прогонки}
Данный метод пригоден только для систем с коэффициентами, удовлетворяющими следующим условиям:
\begin{enumerate}
  \item \(|A_k| + |C_k| \leq |B_k|,k=1,...N\)
  \item Ненулевые элементы на главной диагонали
\end{enumerate}
Алгоритм состоит из двух шагов: прямой и обратных ход:

\subsubsection{Прямой ход}
Прямым ходом называется вычисление прогоночных коэффициентов
\begin{enumerate}
  \item Выразим \(U_1\) из \ref{eq:1_1} \label{eq:2_1}
        \[U_1=-\frac{C_1}{B_1}U_2+\frac{F_1}{B_1}\]
  \item Пусть \label{eq:2_2}
        \[\alpha_1=-\frac{C_1}{B_1}, \beta_1=\frac{F_1}{B_1}\]
  \item Тогда соотношение из П.\ref{eq:2_1} изменится следующим образом: \label{eq:2_3}
        \[U_1=\alpha_2U_2+\beta_2\]
  \item Предположим, что \(U_{k-1}=\alpha_k U_k + \beta_k\) \label{eq:2_4}
  \item Уберём из \(k\)-го уравнения системы \ref{eq:1_1} \(U_{k-1}\), подставив соотношение из П.\ref{eq:2_4} в \(k\)-е уравнение
        \[A_k(\alpha_k U_k + \beta_k) + B_k U_k + C_kU_{k+1}=F_k\]
        Отсюда выразим \(U_k\):
        \[U_k = - \frac{C_k U_{k+1}}{B_k+A_k\alpha_k} + \frac{F_k-A_k\beta_k}{B_k+A_k\alpha_k}\]
        Теперь введём обозначения: \label{eq:2_5}
        \[\beta_{k+1}=\frac{F_k-A_k\beta_k}{B_k+A_k\alpha_k}\]
        \[\alpha_{k+1}= - \frac{C_k}{B_k+A_k\alpha_k}\]
  \item В результате \(k\)-е уравнение сведётся к виду \(U_k=\alpha_{k+1} U_{k+1} + \beta_{k+1}\) \label{eq:2_6}
\end{enumerate}
Отсюда видно, что уравнение системы \ref{eq:1_1} для каждого \(k=1,...,N-1\) приводится к виду П.\ref{eq:2_6}.

Прогоночные коэффициенты \(\alpha_{k+1}, \beta_{k+1}\) вычисляются по формулам из П.\ref{eq:2_5}.

\subsubsection{Обратный ход}
Обратным ходом называется вычисление \(U_k\)
\(N\)-е уравнение системы \ref{eq:1_1} не участвовало на этапе прямого хода.
Пусть \(k=N-1\). Того оно вместе с соотношением из П.\ref{eq:2_6} образует систему:
\[
  \begin{cases}
    A_N U_{N-1} + B_N U_N = F_N,
    \\
    U_{N-1}=\alpha_N U_N + \beta_N
  \end{cases}
\]
Используем эту систему для определения \(U_N\):
\[U_N = \frac{F_N-A_N\beta_N}{A_N\alpha_N+B_N}\]

Остальные \(U_k\) при \(k=N-1,...,1\) считаем по формуле из П.\ref{eq:2_6}.
