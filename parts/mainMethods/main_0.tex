\subsection{Постановка задачи}
Решение системы линейных алгебраических уравнений специального вида:
\[A_kU_{k-1} + B_kU_k + C_kU_{k+1} = F_k,  k=1,…,N , где A_1=C_N=0~(2.1) \]\label{eq:1_1}
\(A_k, B_k, C_k, k=1,…, N,\) - заданные коэффициенты системы, которые можно рассматривать как три диагонали матрицы системы, а остальные коэффициенты системы равны нулю;
\(F_k, k=1,…, N\)-правые части, \(U_k, k=1,…, N\)-искомые значения, решения данной системы. К системам вида (\ref{eq:1_1}) обычно приходят при решении одномерной краевой задачи
для дифференциального уравнения второго порядка, одномерной задачи нестационарной тепловодности или диффузии, многомерных параболических уравнений, локально-нелинейных
одномерных задача и т.д.
